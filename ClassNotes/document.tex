\documentclass{article}
\usepackage{graphicx}
\usepackage{listings}
\usepackage{color}
\usepackage{amsmath}
\usepackage{xcolor}

% These define the code colors when 
% inputing 
%%%%%%%%%%%%%%%%%%%%%%%%%%%%%

\definecolor{dkgreen}{rgb}{0,0.6,0}
\definecolor{gray}{rgb}{0.5,0.5,0.5}
\definecolor{mauve}{rgb}{0.58,0,0.82}
%%%%%%%%%%%%%%%%%%%%%%%%%%%%%
% These define the paramaters for the code

\lstset{frame=tb,
language=Java,
aboveskip=3mm,
belowskip=3mm,
showstringspaces=false,
columns=flexible,
basicstyle={\small\ttfamily},
numbers=none,
numberstyle=\tiny\color{gray},
keywordstyle=\color{blue},
commentstyle=\color{dkgreen},
stringstyle=\color{mauve},
breaklines=true,
breakatwhitespace=true,
tabsize=3
}

%%%%%%%%%%%%%%%%%%%%%%%%%%%%%

\begin{document}

\title{Object Oriented Programming}
\author{Kaleb Moreno}
\date{\today}
\maketitle
\pagebreak
%TOC
\tableofcontents
\pagebreak
%%%%%%%%%%%%%%%%%%%%%%%%%%%%%

\section{Point Classes}
\section{Inheritance}
\paragraph{Major points}

\begin{itemize}
  \item We started with an employee program that demonstrates the common
  characteristics among them. When you inherit you use the keyword
  \textbf{'extends'} as in the example below.
\end{itemize}

\begin{lstlisting}
public class secretary extends Employee {
	// This is where code goes
}
\end{lstlisting}

\begin{itemize}
  \item This allows you to call the \textbf{super class} which houses the parent
  methods and fields.
  \item The following example makes use of the keyword \textbf{super.}
\end{itemize}

\begin{lstlisting}
public class legalsecretary extends Secretary {
	public double Salary() {
		super.getSalary();
	}
}
\end{lstlisting}

\section{Polymorphism}
\paragraph{Major Points}

\begin{itemize}
  \item The word literally means \textbf{many forms} and it is tied into inheritance.
  \item This process refers to the illusion that something is one thing, but it
  is actually something different.
  \item Below is an example: 
\end{itemize}

\begin{lstlisting}
  Employee ed = new Lawyer();
  // You can call any method from the Employee class on ed
  // When a method called on ed it behaves as a Lawyer
  System.out.println(ed.getSalary()); // 50000
  System.out.println(ed.getVacationForm()); // Pink
\end{lstlisting}

\begin{itemize}
  \item You can also pass a subtype of a parameter's type such as in the example
  below:
\end{itemize}

\begin{lstlisting}
  Public class EmployeeMain2 {
    public static void main(String[] theArgs) {
      Emlpoyee[] e = [new Lawyer(), new Secretary(), new Marketer(), new Legalsecretary() ];

      for (int i = o; i < e.length; i++) {
        System.out.println("salary: " + e[i].getSalary());
        System.out.println("v.days: " + e[i].getVacationDays());
        // The rest of this code was missed.. 
      }
    }
  }
\end{lstlisting}

\section{Type Casting}
\paragraph{Major Points}

\begin{itemize}
  \item This comes in handy when you want to be able to compare all objects 
  \item the keyword \textbf{instanceof} is used in this case 
  \item This is the only method that can have multiple return statements
  \item This is a tool you can use to get things past the compiler, but it may
  cause errors
  \item Below is code where \textbf{instanceof} can be seen in use as well as
  \textbf{Point}:
\end{itemize}

\begin{lstlisting}
  public boolean equals(Object 0) {
    if (o instanceof Point) {
      // Notice the cast to a Point
      Point other = (Point) o;
      // Notice the multiple return statements
      return x == other.x && y == other.y;
    } else {
      return false;
    }
  }
\end{lstlisting}
\pagebreak
\section{Interfaces}
\paragraph{Major Points}

\begin{itemize}
  \item The primary purpose of an interface is to set a specific behavior to
  all the classes that reference it. 
  \item The keyword that is used for including an interface in a class
  is\textbf{implements} and its is done in the class header, after the name of
  the class.
\end{itemize}

\begin{lstlisting}
  // This is an example of what might be in a class when implementing an interface..
  // Remember you have to have all the methods that shape is looking for.
  public class Circle implements Shape {
  private double radius;
}
\end{lstlisting}

\begin{itemize}
  \item When you want to create an interface you essentially create a class that
  doesn't have any defined methods.
  \item The code below illustrates:
\end{itemize}

\begin{lstlisting}
  public interface { 
    public type name(type name..., type name);
    public type name(type name..., type name);
  }
  // Notice the lack of code defining the actual method
  // This is to be done in the class that implements this interface
  public interface Vehicle {
    public double speed();
  }
\end{lstlisting}

\section{Pre-Topic Notes}
\paragraph{Major Points}

\begin{itemize}
  \item Keep in mind overloaded methods and constructors..Those jacked me up
  \item Casting different classes doesn not change the class that is called, but
  it can make the compiler do things you want. 
  \item \textcolor{red}{The suggested exercises:} Are very important  
  \item Abstract classes can have anything in it, but it has to have one
  abstract method to be considered an abstract method
  \item Mice should move randomly and not together in the next project!
  \item Look at comparable class in the end of the slides
\end{itemize}

\subsection{Class program that we went over}

\begin{itemize}
  \item Constants in the interface, make things easier to manage instead of
  putting them directly in the class.
  \item Abstract classes can pass responsibility down to lower classes
  \item \textcolor{red}{break statements} can be used in switch statements
  \textbf{only}
  \item 
\end{itemize}

\section{ArrayLists}
\paragraph{Major Points}

\begin{itemize}
  \item The positions change if something is deleted, and incrementing can be an
  issue.
  \item 
\end{itemize}

\section{Notes From meeting with Prof 2/7}

\paragraph{Program project 3 corrections}

\begin{itemize}
  \item No space between urinary operators 
  \item Make sure you put a multiline comment on the top of the file
  \item Every Field should have javadoc 
\end{itemize}

\section{The For Each Loop}
\paragraph{This works just like the for loop in python and does things}

\begin{itemize}
  \item Instead of using the standard for loop for iterating through ArrayLists, you can use this type of stucture to save some code.
\end{itemize}

\begin{lstlisting}
for (type value: arrayname) {
	//Do some stuff here 
}
\end{lstlisting}

\section{Collections: LinkedLists / ArrayLists}
\subsection{LinkedLists}

\begin{itemize}
  \item Make sure you know the difference between the two, and when to use them.
  \item Applications that have a lot of coming and goind, a LinkedList would be much better and faster.
\end{itemize}

\subsection{Collection Interface}

\begin{itemize}
  \item There are a lot of methods that you can use such as:
  \item add, addAll, Clear, Contains, containsAll
\end{itemize}


























\end{document}
