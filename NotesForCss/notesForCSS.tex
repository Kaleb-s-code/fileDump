\documentclass{article}
\usepackage{graphicx}
\usepackage{listings}
\usepackage{color}

\definecolor{dkgreen}{rgb}{0,0.6,0}
\definecolor{gray}{rgb}{0.5,0.5,0.5}
\definecolor{mauve}{rgb}{0.58,0,0.82}

\lstset{frame=tb,
language=Java,
aboveskip=3mm,
belowskip=3mm,
showstringspaces=false,
columns=flexible,
basicstyle={\small\ttfamily},
numbers=none,
numberstyle=\tiny\color{gray},
keywordstyle=\color{blue},
commentstyle=\color{dkgreen},
stringstyle=\color{mauve},
breaklines=true,
breakatwhitespace=true,
tabsize=3
}
\begin{document}


\title{Object Oriented}
\maketitle

\section{Covered writing the program }

\subsection{Files Used:}

\begin{enumerate}
\item{Day4Code \& Lab} 
\end{enumerate}

\subsection{Notes}
I never downloaded the files for this, so I can't compile yet. \\
\%5d is the format specifier that should make things align

\begin{lstlisting}
public class CheckArrays {
public static boolean sameDimensions(int[][] theA, int[][]theB) {
    return theA.length == theB.length && theA[0].length == theB[0].length;
}
public static boolean areEqual(int[][]theA, int[][]theB) {
    boolean result = sameDimensions(theA, theB);
    for (int row = 0; row < theA.length && result; row ++) {
        for (int col = 0; col < theA[row] && result; col ++) {
            result = theA[row][col] == theB[row][col];
        }
    }
return result;
}
}
public static int sum(int[][]theA, int[][]theB) {
int result = 0;
for (int row = 0; row < theA.length; row ++) {
    for (int col = 0; col < theA[row]; col ++) {
        result += theA[row][col];
    }
}
return result;
}
\end{lstlisting}

\section{Day 5 Code and notes}
\subsubsection{ Misc Notes}
This is the code for the day explaining a bomb program that uses point classes and final parameters. These points make use of implied parameters,and there are \textbf{this} code references?? \\


if throwing exceptions, do that fist in a method.\\

\begin{enumerate}
\item Protecting methods is one of the key points in this program
\item reuseability is also key as demonstrated in this program
\end{enumerate}




\begin{lstlisting}
package app;

public final class Point {
    public static void main(String[] args) throws Exception {
        public static final int DEFAULT_X = 0;
        public static final int DEFAULT_Y = 0;
        private int myX;
        private int myY;
        public Point(final int theX, final int theY) {
            if (theX < 0 || theY < 0) {
                throw new IllegalArgumentException("Coordinates cannot " + "be negative.");
            }
            myX = theX;
            myY = theY;
        }
        public Point() {
            this(DEFAULT_X, DEFAULT_Y);
        }
        public Point(Point theP) {
            this(theP.myX, theP.myY);
        }
        public int getX() {
            return myX;
        }
        public int getY() {
            return myY;
        }
    
        public double calculateDistance(final Point theOtherPoint) {
            if (theOtherPoint == null) {
                throw new NullPointerException ("Cannot use a point of null" + "to calculate a distance");
            }
            final double dx = myX - theOtherPoint.myX;
            final double dy = myY - theOtherPoint.myY;
            return Math.sqrt(dx * dx + dy * dy);
    
        }

        public void setX(final int theX) {
            if (theX < 0) {
                throw new IllegalArgumentException("Coordinates cannot " + "be negative.");
            }
            // This is called a mutator method 
            myX = theX;
        }
        public void setY(final int theY) {
            if (theY < 0) {
                throw new IllegalArgumentException("Coordinates cannot " + "be negative.");
            }
            // This is called a mutator method 
            myY = theY;
        }
    }
}
public final class Point {
    public static void main(String[] args) throws Exception {
        public static final int DEFAULT_X = 0;
        public static final int DEFAULT_Y = 0;
        private int myX;
        private int myY;
        public Point(final int theX, final int theY) {
            if (theX < 0 || theY < 0) {
                throw new IllegalArgumentException("Coordinates cannot " + "be negative.");
            }
            myX = theX;
            myY = theY;
        }
        public Point() {
            this(DEFAULT_X, DEFAULT_Y);
        }
        public Point(Point theP) {
            this(theP.myX, theP.myY);
        }
        public int getX() {
            return myX;
        }
        public int getY() {
            return myY;
        }
        public double calculateDistance(final Point theOtherPoint) {
            if (theOtherPoint == null) {
                throw new NullPointerException ("Cannot use a point of null" + "to calculate a distance");
            }
            final double dx = myX - theOtherPoint.myX;
            final double dy = myY - theOtherPoint.myY;
            return Math.sqrt(dx * dx + dy * dy);
    
        }

        public void setX(final int theX) {
            if (theX < 0) {
                throw new IllegalArgumentException("Coordinates cannot " + "be negative.");
            }
            // This is called a mutator method which change state of object 
            myX = theX;
        }
        public void setLocation(int theX, int theY) {
            if (theX < 0 || theY < 0) {
                throw new IllegalArgumentException("Coordinates cannot " + "be negative.");
            }
            // This is called a mutator method which change state of object
            myX = theX;
            myY = theY;
        }
        public void translate(int theX, int theY) {
            if (theX < 0 || theY < 0) {
                throw new IllegalArgumentException("Coordinates cannot " + "be negative.");
            }
            // This is called a mutator method which change state of object
            setLocation(myX + theX, myY + theY);
        }

        public String toString(){
            String result = "";
            result += "Point";
            result += "(";
            result += myX;
            result += ", " + myY + ")";
            return result; 
    
        }
    }    
    
}
\end{lstlisting}

This is another class held in a different file that deals with the 
checkpoints.
\begin{lstlisting}
// This is a a different class that we made in a differnet file that we used to execute the point class and make the points
public class Checkpoint {
    public static void main(String[] theArgs) {
        Point p1 = new Point(s, 9);
        Point p2 = new Point();
        Point p3 = new Point(p1);
        // Note that here you don't have to do p1.toString()
        // Becasue it is implicitly understood as that 
        System.out.println(p1 + "\n" + p2 +"\n" + p3);
        p1.setLocation()

    }
}

\end{lstlisting}

\section{1/29/2019}
\subsubsection{Misc}
Files used:

\begin{enumerate}
\item codeFromClass/MyClass.java
\end{enumerate}

\subsection{Inheritance}
We started an employee program that demonstrated common characteristics among them.\\
when you Inherite, you do:

\begin{lstlisting}
public class secretary extends Employee {
// This is where the method goes
}
\end{lstlisting}

You can override to a new method in the subclass\\
When you need to access base pay for example, you should call the super class:

\pagebreak

\begin{lstlisting}
public class Legalsecretary extends Secretary {
  public double Salary() {
  // The keyword super grabs from the super class the base pay for example  
  super.getSalary();
  }  
}
\end{lstlisting}

You can call the superclasses constructors by doing this:

\begin{lstlisting}
  super(years) // Calls the Employee constructor 
\end{lstlisting}

\subsubsection{Next programming project}
Nevermind the crazy driver it just tests the methods\\
Do not write it based on the output, or you'll be jacked up. Its meant to be confusing\\
Makes use of interfacing which describe the methods that you have to include in your class\\
Needs javadoc for every class and even the constructor 

\subsection{Polymorphism}
Every class is an objectm




\end{document}