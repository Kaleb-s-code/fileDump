\documentclass{article}
\usepackage{graphicx}
\usepackage{listings}
\usepackage{color}

\definecolor{dkgreen}{rgb}{0,0.6,0}
\definecolor{gray}{rgb}{0.5,0.5,0.5}
\definecolor{mauve}{rgb}{0.58,0,0.82}

\lstset{frame=tb,
language=Java,
aboveskip=3mm,
belowskip=3mm,
showstringspaces=false,
columns=flexible,
basicstyle={\small\ttfamily},
numbers=none,
numberstyle=\tiny\color{gray},
keywordstyle=\color{blue},
commentstyle=\color{dkgreen},
stringstyle=\color{mauve},
breaklines=true,
breakatwhitespace=true,
tabsize=3
}
\begin{document}


\title{Notes for 1/22}
\author{Kaleb Moreno}
\maketitle

\section{Covered writing the program }

\subsection{Files Used:}

\begin{enumerate}
\item{Day4Code \& Lab} 
\end{enumerate}

\subsection{Notes}
$\rightarrow$I never downloaded the files for this, so I can't compile yet. \\

$\rightarrow$\%5d is the format specifier that should make things align
\begin{lstlisting}
    public class CheckArrays {
        public static boolean sameDimensions(int[][] theA, int[][]theB) {
            return theA.length == theB.length && theA[0].length == theB[0].length;
        }
        public static boolean areEqual(int[][]theA, int[][]theB) {
            boolean result = sameDimensions(theA, theB);
            for (int row = 0; row < theA.length && result; row ++) {
                for (int col = 0; col < theA[row] && result; col ++) {
                    result = theA[row][col] == theB[row][col];
                }
            }
        return result;
        }
    }
    public static int sum(int[][]theA, int[][]theB) {
        int result = 0;
        for (int row = 0; row < theA.length; row ++) {
            for (int col = 0; col < theA[row]; col ++) {
                result += theA[row][col];
            }
        }
    return result;
    }
\end{lstlisting}

\end{document}